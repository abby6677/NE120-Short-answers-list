\documentclass{article}
\usepackage[utf8]{inputenc}
\usepackage[left=0.5 in,top=0.5in,right=0.5 in,bottom=0.5in]{geometry}
\usepackage{graphicx}
\usepackage{amsmath}

\begin{document}

\large{Chapter 1 - Introduction}
\begin{enumerate}
    \item A PWR has
    \emph{
    \begin{itemize}
        \item Higher pressure than BWR
        \item 3 water loops
        \item Control rods on top
    \end{itemize}}
    \item Zirconium is used in LWR application because
    \emph{
    \begin{itemize}
        \item It is corrosion resistant under PWR condition
        \item It has decent strength (It is not ductile)
        \item It has a low neutron cross section
    \end{itemize}
    }
    \emph{Note that it does react with water at high temperature.}
    \item Non MOX LWR nuclear fuel is made from 
    \emph{
    \begin{itemize}
        \item Oxygen
        \item Uranium
    \end{itemize}
    }
    \emph{Note that Plutonium only shows up in MOX fuel}
    \item Steel is 
    \emph{
    \begin{itemize}
        \item An alloy of Fe-C
        \item Fe-Cr-Ni Alloy
    \end{itemize}
    }
    \emph{Not pure Fe}
    \item What temperature does a LWR operate at?
    \emph{\begin{itemize}
        \item $\approx300^\circ C$
    \end{itemize}}
    \item Are material property changes a function of both dose and temperature?
    \emph{\begin{itemize}
        \item Yes
    \end{itemize}}
\end{enumerate}
\newpage
\large{Chapter 2 - Crystal Structures and Defects}
\begin{enumerate}
    \item Which crystal structures are the most densely packed?
    \emph{
    \begin{itemize}
        \item HCP
        \item FCC
    \end{itemize}
    }
    \emph{Both has APF = 0.74}
    \item Which statements describe the orthorhombic crystal?
    \emph{
    \begin{itemize}
        \item $a\neq b\neq c$
        \item $\alpha=\beta=\gamma$
    \end{itemize}
    }
    \item How many atoms are in a FCC unit cell?
    \emph{\begin{itemize}
        \item 4
    \end{itemize}}
    \item Zircalloy 4 is 
    \emph{\begin{itemize}
        \item HCP
    \end{itemize}}
    \emph{This is why Zircalloy is brittle.}
    \item Vacancies are 
    \emph{\begin{itemize}
        \item 0 dimensional defects
    \end{itemize}}
    \item Dislocations are
    \emph{\begin{itemize}
        \item 2 dimensional defects
    \end{itemize}}
    \item Do all refractory metals have an HCP lattice?
    \emph{\begin{itemize}
        \item False
    \end{itemize}}
    \emph{Refractory metals are a group of metals that are highly resistant to heat and wear. They are mostly BCC.}
    \item Does CsCl and NaCl have the same crystal structure?
    \emph{\begin{itemize}
        \item No
    \end{itemize}}
    \item FeS in steel is
    \emph{\begin{itemize}
        \item Inclusion
        \item Precipitate
        \item A second phase
    \end{itemize}}
    \emph{Not an interstitial}
    \item A Frenkel Pair is 
    \emph{\begin{itemize}
        \item A vacancy and an interstitial
    \end{itemize}}
    \item Defects are to be avoided in materials since they decrease the material's property.
    \emph{\begin{itemize}
        \item False
    \end{itemize}}
    \item Hume Rothery Rules tell us 
    \emph{\begin{itemize}
        \item If an element is soluble in another element.
    \end{itemize}}
    \item What is the closed packed plane in BCC?
    \emph{\begin{itemize}
        \item (110)
    \end{itemize}}
    \item What is the closed packed plane in FCC?
    \emph{\begin{itemize}
        \item (111)
    \end{itemize}}
    \item What vacancy sites exists in a cubic crystal?
    \emph{\begin{itemize}
        \item Octahedron sites
        \item Tetrahedron sites
    \end{itemize}}
    \emph{Octahedron sites are always preferred, no matter in FCC or BCC or how large the sites actually are.}
    \item How are vacancies characterized?
    \emph{\begin{itemize}
        \item Positron Spectroscopy
    \end{itemize}}
    \item Do we have more vacancies or interstitials in an equilibrium material?
    \emph{\begin{itemize}
        \item Vacancies
    \end{itemize}}
    \item Where do self-interstitials sit?
    \emph{\begin{itemize}
        \item Dumbbell sites
    \end{itemize}}
    \item Which Fe structure contains more C?
    \emph{\begin{itemize}
        \item FCC
    \end{itemize}}
    \emph{As discussed Fe-C phase diagrams, FCC iron, or austenite, can contain a higher C concentration. Note that even though FCC has a higher APF, it also has larger interstitial sites.}
    \item Plotting log property against 1/T is called
    \emph{\begin{itemize}
        \item Arrhenius plot
    \end{itemize}}
    \item What is the number of slip systems in FCC?
    \emph{\begin{itemize}
        \item $\text{4 Planes}\times\text{3 Directions}=12$ 
    \end{itemize}}
    \item Can BCC crystals activate other slip systems than the [110]<111>?
    \emph{\begin{itemize}
        \item Yes
    \end{itemize}}
    \item Can a dislocation line just stop in the crystal?
    \emph{\begin{itemize}
        \item No
    \end{itemize}}
    \emph{It can end at grain boundaries, loop on itself, or end on each other}
    \item Does a burgers vector has the same value as the lattice constant in BCC?
    \emph{\begin{itemize}
        \item No
    \end{itemize}}
    \item What is the unit of dislocation density?
    \emph{\begin{itemize}
        \item $\#/m^2$
    \end{itemize}}
    \item Are interstitials attracted to a compression stress field or a tension stress field?
    \emph{\begin{itemize}
        \item Tension stress field
    \end{itemize}}
    \item What is a Frank-Read source?
    \emph{\begin{itemize}
        \item It is a mechanism of dislocation generation where, to minimize surface energy, a dislocation extends and loops on itself. 
    \end{itemize}}
    \item Can interaction of dislocations with other defects lead to a multiplication of dislocations?
    \emph{\begin{itemize}
        \item Yes
    \end{itemize}}
    \item What is a stacking fault?
    \emph{\begin{itemize}
        \item When a specific stacking order is changed to a different order by removing or adding a lattice plane.
    \end{itemize}}
    \item What happens when two dislocations gliding on different lattice planes meet?
    \emph{\begin{itemize}
        \item They interact and form jogs (2 screw dislocations) and kinks (2 edge dislocations) and become less mobile.
    \end{itemize}}
    \item Does large $\Sigma$-number grain boundary mean more or less misorientation between the two grains?
    \emph{\begin{itemize}
        \item More misorientation
    \end{itemize}}
    \item Does fast cooling from the melt foster smaller or larger grains?
    \emph{\begin{itemize}
        \item Smaller
    \end{itemize}}
    \emph{More nucleation, smaller grains.}
    \item If many dislocations arrange themselves on top of each other, what kind of grain boundary do they create?
    \emph{\begin{itemize}
        \item Small angle grain boundaries
    \end{itemize}}
    \item Generally speaking, does finer grains lead to better mechanical properties?
    \emph{\begin{itemize}
        \item Yes
    \end{itemize}}
    \item A $\Sigma3$ grain boundary is a grain boundary where
    \emph{\begin{itemize}
        \item Every atom matches up 
        \item There is low grain boundary energy
        \item It is a twin boundary
    \end{itemize}}
\end{enumerate}

\newpage

\large{Chapter 3 - Thermodynamics}
\begin{enumerate}
    \item The guiding equation for minimizing the energy in a system is
    \emph{\begin{itemize}
        \item $G=H-TS$
    \end{itemize}}
    \item A dendrite is
    \emph{\begin{itemize}
        \item A specific shape of a solidified material
    \end{itemize}}
    \emph{According to Wikipedia, it s a ``characteristic tree-like structure of crystals growing as molten metal solidifies''.}
    \item Exactly at the melting temperature, is there a preference for having a solid or liquid?
    \emph{\begin{itemize}
        \item There is no preference
    \end{itemize}}
    \item Is undercooling needed to drive solidification? Why?
    \emph{\begin{itemize}
        \item Yes, since the energy needed to form a new surface needs to be overcome.
    \end{itemize}}
    \item Latent heat is
    \emph{\begin{itemize}
        \item The heat or energy that is absorbed or released during a phase change of a substance.
        \item What provides nice cool refreshments until all the ice is melted on a hot summer day :)
    \end{itemize}}
    \item Is entropy higher at higher temperatures?
    \emph{\begin{itemize}
        \item No
    \end{itemize}}
    \item How is entropy calculated?
    \emph{\begin{itemize}
        \item Quantifying the number of distinguishable ways one can arrange A and B atoms.
    \end{itemize}}
    \item The Gibbs free energy of mixing is additional to the ideal Gibbs free energy of two substances A and B?
    \emph{\begin{itemize}
        \item True
    \end{itemize}}
    \item If temperature versus time is plotted for a material at a constant cooling rate, what does the slope of the graph describe?
    \emph{\begin{itemize}
        \item Heat capacity
    \end{itemize}}
    \item Does pressure affect the phases in a material?
    \emph{\begin{itemize}
        \item Yes
    \end{itemize}}
    \item What does on use to alloy Pu with to avoid numerous phase transformations?
    \emph{\begin{itemize}
        \item Ga
    \end{itemize}}
    \item What marks the point when Gibbs free energy of the liquid and solid intersect in a G-T plot?
    \emph{\begin{itemize}
        \item Melting point
        \item Freezing point
    \end{itemize}}
    \item What are the two competing energy terms that govern nucleation of a new crystal?
    \emph{\begin{itemize}
        \item Surface energy
        \item Gibbs free energy
    \end{itemize}}
    \item In a solid solution crystal, is the Gibbs free energy of mixing is lower than the Gibbs free energy of the pure substances?
    \emph{\begin{itemize}
        \item Yes
    \end{itemize}}
    \item How many degrees of freedom do we have at the eutectoid/eutectic point?
    \emph{\begin{itemize}
        \item 0
    \end{itemize}}
    \emph{$F=C-P+1=2-3+1=0$}
    \item Does a newly solidified grain have the same composition throughout the entire diameter?
    \emph{\begin{itemize}
        \item No
    \end{itemize}}
    \item How many phases is pearlite composed of?
    \emph{\begin{itemize}
        \item 2 phases, ferrite and cementite  
    \end{itemize}}
    \item What factors might determine which phases of an alloy are present?
    \emph{\begin{itemize}
        \item Temperature
        \item Enthalpy of mixing
        \item Cooling time
    \end{itemize}}
    \item According to Gibbs phase rule, how many degrees of freedom does a single phase regime in a two element phase diagram have?
    \emph{\begin{itemize}
        \item 2
    \end{itemize}}
    \emph{$F=C-P+1=2-1+1=2$}
    \item What is the eutectic microstructure of steel?
    \emph{\begin{itemize}
        \item Pearlite
    \end{itemize}}
    \item What shape does the G-X curves have at the eutectoid line?
    \emph{\begin{itemize}
        \item $\alpha,\beta,\gamma$ all sit on the same common tangent
    \end{itemize}}
    \item Is the martensitic phase transformation without diffusion?
    \emph{\begin{itemize}
        \item Yes
    \end{itemize}}
    \emph{Twins and/or laths show up in martensites.}
    \item Shape memory alloys
    \emph{\begin{itemize}
        \item Change their shape due to phase transformations
    \end{itemize}}
    \item What microsturctures and phases form based on cooling rates can be seen in a 
    \emph{\begin{itemize}
        \item Time Temperature Transformation (TTT) graph
    \end{itemize}}
    \item A heat treatment is 
    \emph{\begin{itemize}
        \item Changing a materials property by guiding heating and cooling cycles
        \item Using phase transformations to change materials' properties
        \item Using phase transformations to change materials' shapes
    \end{itemize}}
    \item The two types of TTT graphs are
    \emph{\begin{itemize}
        \item Continuous cooling
        \item Isothermal 
    \end{itemize}}
    \item A test to obtain all possible cooling rates on one sample is called
    \emph{\begin{itemize}
        \item The Jominy test
    \end{itemize}}
    \item Heat treatments on metals are performed to 
    \emph{\begin{itemize}
        \item Change the material's mechanical properties
        \item Change the material's microstructure
        \item Change a material's response to radiation
    \end{itemize}}
    \emph{not to change its density, chemistry, or thermal properties}
    \item Austenite (FCC) stabilizers are 
    \emph{\begin{itemize}
        \item Ni, Co, Mn
    \end{itemize}}
    \item Ferrite (BCC) stabilizers are
    \emph{\begin{itemize}
        \item Cr, Al, Ti, Si, Mo, W, V
    \end{itemize}}
     \item What does the ternary phase diagram display at its edges?
    \emph{\begin{itemize}
        \item The binary phase diagrams
    \end{itemize}}
    \item An isothermal cross section in a ternary phase diagram is 
    \emph{\begin{itemize}
        \item A plot showing all phases at a set temperature present at that time.
    \end{itemize}}
    \item Can a ternary eutectic have a melting temperature below the binary eutectic?
    \emph{\begin{itemize}
        \item Yes
    \end{itemize}}
    \item What treatment is needed to get hardened steel?
    \emph{\begin{itemize}
        \item Fast cooling (Quenching)
    \end{itemize}}
    \item What heat treatment is needed to recover some of the toughness?
    \emph{\begin{itemize}
        \item Tempering
    \end{itemize}}
    \item What heat treatment did a tempered martensitic steel undergo?
    \emph{\begin{itemize}
        \item Heating to austenite $\Rightarrow$ Quench $\Rightarrow$ Heating to below austenite 
    \end{itemize}}
    \item Why are some austenitic stainless steel magnetic?
    \emph{\begin{itemize}
        \item They contain remaining ferrite
    \end{itemize}}
\end{enumerate}

\newpage

\large{Chapter 4 - Diffusion}
\begin{enumerate}
    \item Diffusion mechanisms from fastest to slowest are:
    \emph{\begin{itemize}
        \item Surface $>$ Grain Boundary $>$ Lattice
    \end{itemize}}
    \item The following terms are part of the diffusion equation:
    \emph{\begin{itemize}
        \item Energy it takes to form an vacancy
        \item Energy it takes to move an atom to a vacant lattice site
        \item Likelihood of the jump happening in the direction of a vacancy site
    \end{itemize}}
    \item Fick's 2nd law is an extension to time from Fick's 1st law
    \emph{\begin{itemize}
        \item True
    \end{itemize}}
    \emph{Fick's 1st law is for steady state, 2nd law is for nonsteady state.}
    \item Is the diffusion coefficient is somehow related to melting temperature?
    \emph{\begin{itemize}
        \item Yes
    \end{itemize}}
    \emph{The activation energy is proportional to the melting temperature.}
    \item Kirkendall pores are
    \emph{\begin{itemize}
        \item Holes that form because the diffusion of two elements in each other can have different constants.
    \end{itemize}}
    \item Why is Cu in RPVs (Reactor Pressure Vessel) a problem?
    \emph{\begin{itemize}
        \item Cu precipitates in service over a long period of time
    \end{itemize}}
    \item What can part of the diffusion problem be approximated with?
    \emph{\begin{itemize}
        \item A parabolic function
    \end{itemize}}
    \item What kind of function can the diffusion equation be solved with?
    \emph{\begin{itemize}
        \item An error function
    \end{itemize}}
    \item Does diffusion coefficients depend on crystal structure?
    \emph{\begin{itemize}
        \item Yes
    \end{itemize}}
    \item Doping nuclear fuel with ions of a different charge state leads to 
    \emph{\begin{itemize}
        \item Change of the diffusion coefficients
        \item Change of the microstructure
    \end{itemize}}
    \emph{It does not lead to the change of fracture toughness of the material}
    \item What two methods exist for setting the boundary condition on a diffusion problem?
    \emph{\begin{itemize}
        \item $C_0$ is constant
        \item $(C_0-C_1)/2$ is constant
    \end{itemize}}
\end{enumerate}

\newpage

\large{Chapter 5 - Mechanical Properties}
\begin{enumerate}
    \item Yield strength marks the 
    \emph{\begin{itemize}
        \item Highest stress in a stress-strain curve (10/6 ???)
        \item Transition from elastic to plastic behavior
        \item Start of the movement of dislocations
    \end{itemize}}
    \item Does the elastic component need to be subtracted when quantifying the plastic strain in a stress strain curve?
    \emph{\begin{itemize}
        \item Yes
    \end{itemize}}
    \item When calculating the stress in a tube wall in MPa, can we just use the pressure inside of the tube in MPa?
    \emph{\begin{itemize}
        \item No
    \end{itemize}}
    \item Is there quantitative difference between a true and engineering stress strain curve?
    \emph{\begin{itemize}
        \item Yes
    \end{itemize}}
    \item What other properties is the elastic modulus related to?
    \emph{\begin{itemize}
        \item Thermal expansion
        \item Melting temperature
    \end{itemize}}
    \emph{not density or yield strength}
    \item Is the elastic modulus isotropic?
    \emph{\begin{itemize}
        \item No, depends on crystallographic orientation. 
    \end{itemize}}
    \item What is the Poison's ratio for most crystal structures?
    \emph{\begin{itemize}
        \item 0.3
    \end{itemize}}
    \item What is the elastic modulus for most steels?
    \emph{\begin{itemize}
        \item 200GPa
    \end{itemize}}
    \item Is there a rule of thumb to correlate hardness to yield stress?
    \emph{\begin{itemize}
        \item Yes
    \end{itemize}}
    \emph{Tensile stress (MPa) $=3.45\times$ Brinell hardness}
    \item Does hardness measure force/area?
    \emph{\begin{itemize}
        \item Yes
    \end{itemize}}
    \item What is the Schmid factor?
    \emph{\begin{itemize}
        \item It describes the correlation between the applied stress and the shear stress on the slip system.
    \end{itemize}}
    \item Is plastic deformation based on dislocation based on shear stress?
    \emph{\begin{itemize}
        \item Yes
    \end{itemize}}
    \item If a material is constraint but heated up leading to a thermal expansion, does the material experience stress? 
    \emph{\begin{itemize}
        \item Yes
    \end{itemize}}
    \item Can the strain from thermal expansion be used to calculate the stress with Hook's law?
    \emph{\begin{itemize}
        \item Yes
    \end{itemize}}
    \item Can thermal expansions cause stresses exceeding yield stress?
    \emph{\begin{itemize}
        \item Yes
    \end{itemize}}
    \item Texture describes
    \emph{\begin{itemize}
        \item Grains being aligned in a specific crystallographic direction
    \end{itemize}}
    \item A pole figure shows
    \emph{\begin{itemize}
        \item The number of grains oriented in a specific way in respect to the X-ray beam
    \end{itemize}}
    \item Pressurized pipes and vessels always rupture in 
    \emph{\begin{itemize}
        \item The length axis  
    \end{itemize}}
    \emph{The azimuthal stress is double the axial stress}
    \item In nuclear power plants, there are both high cycle fatigue and low cycle fatigue?
    \emph{\begin{itemize}
        \item True
    \end{itemize}}
    \item Can fatigue failure happen below yield strength?
    \emph{\begin{itemize}
        \item  Yes
    \end{itemize}}
    \item The Schmid factor describes
    \emph{\begin{itemize}
        \item  The relationship between an applied stress to the critical resolved shear stress
    \end{itemize}}
    \emph{$\sigma_s=\frac{F}{A}\cos\theta\cos\phi$}
    \item Is the critical resolved shear stress different for different orientations?
    \emph{\begin{itemize}
        \item  No
    \end{itemize}}
    \emph{The Schmid factor takes care of orientation differences}
    \item A Cottrell cloud is
    \emph{\begin{itemize}
        \item  An aglomoration of interstitial defects around a dislocation
    \end{itemize}}
    \item Do finer grains reduce yield strength due to enhanced grain boundary slipping?
    \emph{\begin{itemize}
        \item  No
    \end{itemize}}
    \emph{Finer grains generally give better mechanical properties}
    \item Strain aging
    \emph{\begin{itemize}
        \item  Increases yield stress due to plastic deformation and diffusion of interstitials
    \end{itemize}}
    \item Creep is
    \emph{\begin{itemize}
        \item The slow deformation of materials
        \item A phenomena that occurs over long periods of time at high temperatures
    \end{itemize}}
    \item The three mechanisms of creep are
    \emph{\begin{itemize}
        \item Coble creep
        \item Nabarro-Herring creep
        \item dislocation climb
    \end{itemize}}
    \item Creep is affected by
    \emph{\begin{itemize}
        \item Grain size
        \item Temperature
        \item Time
        \item Stress
    \end{itemize}}
    \item Creep rate is described by
    \emph{\begin{itemize}
        \item  An Arrhenius relationship
    \end{itemize}}
    \item In a polycrystalline material, different grains take different load because
    \emph{\begin{itemize}
        \item  Different orientations have different elastic modulus
    \end{itemize}}
    \item Notches in materials are dangerous because
    \emph{\begin{itemize}
        \item  They cause stress concentrations
    \end{itemize}}
    \item A screw or bolt is meant ot be loaded in 
    \emph{\begin{itemize}
        \item  Tension
    \end{itemize}}
    \item The metal with the lowest thermal expansion coefficient is
    \emph{\begin{itemize}
        \item  Tungsten
    \end{itemize}}
    \emph{The higher the melting point, the lower the thermal expansion coefficient}
    \item A Charpy test is
    \emph{\begin{itemize}
        \item  An experiment assessing a materials energy absorbed upon impact
    \end{itemize}}
    \item The Von Mises stress
    \emph{\begin{itemize}
        \item Takes into account that the hydrostatic stress state does not contribute to the shape change of the material
        \item Allows us to compare a multi-axis stress state to an uniaxial tensile test
    \end{itemize}}
    \item A figure of merit like the thermal stress parameter 
    \emph{\begin{itemize}
        \item  Allows us to balance different material properties to make good materials choices in a simple way
    \end{itemize}}
    \emph{The larger the thermal stress parameter, M, the more thermal stress resistant the material is}
    \item The Larson Miller Parameter
    \emph{\begin{itemize}
        \item  Establishes the effect of temperature and time on creep
    \end{itemize}}
    \emph{does not establish how stress and time relate}
\end{enumerate}

\newpage
\large{Chapter 6 - Atomic collisions \& Defect Production}
\begin{enumerate}
    \item The energy transferred from an incoming particle to a target particle depends on
    \emph{
    \begin{itemize}
        \item Mass of the incoming particle
        \item Mass of the target
        \item Energy of the incoming particle
    \end{itemize}
    }
    \item The displacement energy is in the range of 
    \emph{\begin{itemize}
        \item 10's of eV
    \end{itemize}}
    \item Does nuclear fuel show cracks during service?
    \emph{\begin{itemize}
        \item Yes
    \end{itemize}}
    \item The average energy transfer from an incoming particle to a target is 
    \emph{\begin{itemize}
        \item $T_{ave}=\frac{1}{2}T_{max}=\frac{1}{2}\left(\frac{4mM}{(m+M)^2}\right)E$
    \end{itemize}}
    \item How many displacements do we observe if the energy of the PKA is below $E_d$?
    \emph{\begin{itemize}
        \item 0 displacements
    \end{itemize}}
    \item What does $E_d$ depend on?
    \emph{\begin{itemize}
        \item the crystal direction
    \end{itemize}}
    \emph{NOT the type of material?????????(10/20/2023 Quiz)}
    \item Is the dpa a experimental measureable unit and quantity?
    \emph{\begin{itemize}
        \item No
    \end{itemize}}
    \item What parameters are needed to truly understand the effect of radiation on materials?
    \emph{\begin{itemize}
        \item dpa
        \item dpa rate
        \item Irradiaton temperature
        \item $E_d$
        \item $E_c$
        \item Isotopes produced
    \end{itemize}}
    \item What can happen after a displacement cascade?
    \emph{\begin{itemize}
        \item A vacancy finds another vacancy, an interstitial, or an extended defect, or a grain boundary
        \item An interstitial finds another interstitial or a grain boundary
    \end{itemize}}
    \item Are ions used to emulate neutrons in radiation damage research?
    \emph{\begin{itemize}
        \item Yes
    \end{itemize}}
    \emph{While ion and neutron stopping behaviors are very different, using ion emulations allows the experiment to be done in a very short of time.}
    \item Do fusion neutrons make less damage than fast reactor neutrons?
    \emph{\begin{itemize}
        \item No
    \end{itemize}}
    \item What does nuclear fuel gets mostly damaged by?
    \emph{\begin{itemize}
        \item Fission Products
    \end{itemize}}
    \item Is the Kinchin-Pease model used in modern code?
    \emph{\begin{itemize}
        \item Yes
    \end{itemize}}
    \emph{SRIM's ``Ion Distribuition and Quick Calculation of Damage'' mode uses the Kinchin-Pease model.}
    \item What does SRIM take into account?
    \emph{\begin{itemize}
        \item The material's density
        \item The energy of the incoming particle
        \item The direction of the beam towards the surface 
    \end{itemize}}
    \emph{NOT the crystal structure or the density change during irradiation}
    \item In order to calculate the dpa from SRIM in a Fe-Cr alloy, what columns need to be added?
    \emph{\begin{itemize}
        \item Cr vacancies + Fe vacancies + PKA
    \end{itemize}}
    \item If one would like to compare dpa calculated by MCNP and dpa calculated by SRIM, which of the detailed and quick calculation modes should be used?
    \emph{\begin{itemize}
        \item The quick calculation mode
    \end{itemize}}
    \item Does SRIM consider crystal orientation?
    \emph{\begin{itemize}
        \item No
    \end{itemize}}
    \item Do lighter or heavier ions cause more damage?
    \emph{\begin{itemize}
        \item Heavier ions
    \end{itemize}}
    \item Are dpa calculations unitless?
    \emph{\begin{itemize}
        \item No
    \end{itemize}}
\end{enumerate}
\newpage
\large{Chapter 7 - Nuclear Fuel Behavior}
\begin{enumerate}
    \item What is yellow cake?
    \emph{\begin{itemize}
        \item U oxide
    \end{itemize}}
    \item Is uranium oxide a good thermal conductor?
    \emph{\begin{itemize}
        \item No
    \end{itemize}}
    \item Are centrifuges the only way to enrich U-238?
    \emph{\begin{itemize}
        \item No
    \end{itemize}}
    \item What is the first barrier to prevent fisson product release?
    \emph{\begin{itemize}
        \item The fuel itself
    \end{itemize}}
    \item Is the cost driver and cost limiting factor for a nuclear power plant uranium since it is rare?
    \emph{\begin{itemize}
        \item No
    \end{itemize}}
    \emph{Uranium is approximately as common as Sn}
    \item What are the 5 fission yield groups?
    \emph{\begin{itemize}
        \item Soluble in the fuel (UO$_2$)
        \item Metallic Inclusions
        \item Second oxides
        \item Alkali metal oxides
        \item Noble gases
    \end{itemize}}
    \item Do metallic inclusions reduce swelling?
    \emph{\begin{itemize}
        \item Yes
    \end{itemize}}
    \emph{Metallic inclusions have a higher density}
    \item Do noble gases reduce the swelling since they are in the UO$_2$ matrix?
    \emph{\begin{itemize}
        \item No
    \end{itemize}}
    \item What does the Ellingham diagram show?
    \emph{\begin{itemize}
        \item Free energy of formation of a new complex
    \end{itemize}}
    \item Do we need to track all elements on the periodic table to determine solid fuel swelling?
    \emph{\begin{itemize}
        \item No
    \end{itemize}}
    \item Where does most of the heating in the fuel originate from?
    \emph{\begin{itemize}
        \item Fission products
    \end{itemize}}
    \item What are very high energy ions called?
    \emph{\begin{itemize}
        \item Swift heavy ions
    \end{itemize}}
    \item A Coulomb explosion is 
    \emph{\begin{itemize}
        \item displacements of electrons leading to subsequent repulsion of positive ions displacement damage
        \item a proposed mechanism causing large amounts of local displacements
    \end{itemize}}
    \emph{NOT Coulomb forces causing cracking in a material; these displacements are not enough to form cracks.}
    \item Start from Quiz 11/6/2023
\end{enumerate}
\end{document}


